\documentclass{article}
\usepackage[utf8]{inputenc}
\usepackage{hyperref}
\usepackage[a4paper, left = 3cm, right = 3cm, top =2cm]{geometry}
\title{\textbf{\underline{Proyecto de Matemática Numérica}} \\ \textit{Predicción de crecimiento demográfico.} \\ \textit{Informe \#2}}
\author{Luis Ernesto Serras Rimada \\ Guillermo Cepero García \\ Miguel Vadim Vilariño Pedraza}

\date{\today}
\begin{document}
\maketitle
En nuestro proyecto final de \textit{Matemática Numérica}, nos gustaría trabajar en el problema de predecir el crecimiento
de una población en función del tiempo. Donde se desea abordar la problematica partiendo de unas cuestiones fundamentales:\\
¿Porqué?, ¿qué resultado se desea obtener y con qué objetivos?.\\
Y pues, se tiene en cuenta que constituye un problema de alta significacion en muchos campos, como la demografía, la
epidemiología y la ecología, que puede resultar útil en estudios demográficos, planificación urbana y análisis de recursos
naturales, entre otros. En otras palabras, por ejemplo, conociendo un resultado aproximado que converja lo maximo posible a una solucion real 
del crecimiento poblacional se podria beneficiar enormemente a la direccion y gestion de recursos de nuestro pais.\\
Luego, se plantea trabajar el asunto con un modelo matematico que nos pueda conducir a dicha prediccion, y se considera 
el \textit{\textbf{modelo logístico}}(1), que es ecuación diferencial que describe cómo la tasa de crecimiento de la población (dP/dt) 
cambia con el tamaño de la población (P(t)). Cuando la población es pequeña, la tasa de crecimiento es alta, ya que 
hay muchos recursos disponibles para cada individuo. A medida que la población crece, la tasa de crecimiento disminuye 
porque hay menos recursos disponibles por individuo. Cuando la población alcanza su capacidad de carga (K), la tasa 
de crecimiento se vuelve cero, indicando que la población ha alcanzado un equilibrio sostenible.\\
\\
\textit{Entonces se tiene que:}
\begin{itemize}
    \item (P(t)) es la población en función del tiempo (t).
    \item (r) es la tasa de crecimiento intrínseca de la población.
    \item (K) es la capacidad de carga o tamaño máximo sostenible de la población.
\end{itemize}\\
\textit{\underline{Parámetros a Estimar:}}
\begin{itemize}
    \item (r): Tasa de crecimiento intrínseca de la población. 
    \item (K): Capacidad de carga de la población.
\end{itemize}\\
\textit{\textbf{\underline{Ecuacion Diferencial de Modelo Logístico:}}} $\frac{dP}{dt} = r \cdot P(t)(1 - \frac{P(t)}{K}) $\\\\
\underline{Para la tasa de crecimiento intrinseca tenemos que:} $r = (\frac{N(t)-F(t)}{P(t)})$,\\ 
donde N(t) es la natalidad en funcion del tiempo y F(t) es la mortalidad en funcion del tiempo.\\
Y para obtener la capacidad de carga de la población cubana siguiendo el contexto del modelo logístico y utilizando 
datos reales, se debe seguir un proceso que involucre la recopilación de datos sobre la población cubana y 
los factores que podrían influir en su capacidad de carga. Estos pueden incluir la disponibilidad de recursos 
naturales, la infraestructura, la salud pública, y la economía.\\
Y posteriormente el análisis de esos datos y la interpretación de los resultados(Analisis Exploratorio).\\
Por ejemplo, supongamos que después de nuestro análisis, encontramos que la tasa de crecimiento intrínseca (r) de la población 
cubana es de 0.02 (un 2\% anual) y que la capacidad de carga (K) estimada es de 15 millones de habitantes. 
Esto significaria que, teóricamente, nuestro pais podría soportar hasta 15 millones de personas sin agotar sus recursos 
vitales, siempre y cuando se mantengan las condiciones actuales y se gestionen adecuadamente los recursos.
\\ 
Como idea de solucion se plantea la función logística, que describe cómo la población crece hacia su capacidad de 
carga y luego se estabiliza.\\ 
\underline{La función logística tiene la forma:} $P(t) = \frac{K}{1 + Ae^{-rt}}$ \\
Donde (A) es una constante que depende de las condiciones iniciales de la población. Esta función muestra claramente 
el punto de inflexión, donde la tasa de crecimiento cambia de positiva a negativa, indicando que la población 
ha alcanzado su capacidad de carga y está comenzando a estabilizarse.\\
Para encontrar (A), se requiere conocer el valor inicial de la población, (P(0)), y usarlo junto con (K) y (r) 
para resolver para (A). La condición inicial (P(0)) te da el valor de la población en el momento inicial, antes de 
que comience el crecimiento logístico.\\
\underline{La ecuación para encontrar (A) es:} $P(0) = \frac{K}{1 + Ae^{0}}$\\
Resolviendo para (A), obtenemos: $A = \frac{K}{P(0)} - 1$\\
Una vez que se tenga el valor de (A), se puede sustituir en la solución general de la ecuación diferencial 
logística para obtener la función que describe el crecimiento de la población hacia su capacidad de carga.
Por ejemplo, si se sabe que la población inicial es de 1000 individuos (P(0) = 1000), la capacidad de carga 
es de 10,000 individuos ((K = 10,000)), y la tasa de crecimiento intrínseca es de 0.01 ((r = 0.01)), 
se puede calcular (A) de la siguiente manera:\\
$A = \frac{10,000}{1000} - 1 = 9$\\
Luego, se puede sustituir (A = 9) en la solución general para obtener la función específica que describe cómo la 
población crecerá hacia su capacidad de carga máxima.\\\\
Posteriormente, como idea de metodo numerico se evalua la alternativa de emplear una regresion lineal con los datos
historicos de poblacion ya recopilados y los parametros (r) y (K) estimados, aunque la regresión lineal simple 
no es directamente aplicable al modelo logístico debido a su naturaleza no lineal, podemos intentar aproximar el 
comportamiento inicial del modelo logístico con una línea recta para fines predictivos iniciales. Esto implica 
ajustar un modelo de regresión lineal entre el tamaño de la población y el tiempo, pero teniendo en cuenta que esta 
aproximación solo será precisa durante los primeros años después de la observación inicial.\\
Por otro lado, tambien tambien se encuentra el modelo exponencial modificado, el cual es una variante del modelo 
exponencial que incorpora un factor de corrección para ajustarse mejor a situaciones donde el crecimiento no es 
constante sino que cambia con el tiempo. Este modelo puede ser particularmente útil cuando el patrón de crecimiento 
es similar a una línea recta pero con variaciones significativas en la tasa de crecimiento a lo largo del tiempo.(\textbf{*posibilidad aun no desarrollada en profundidad*})
\\
Y para la aproximacion numerica se propone usar los datos estimados para resolver la ecuación diferencial del modelo 
logístico numéricamente. Para hacer esto, podemos utilizar métodos numéricos como el método de \textnf{\textit{Euler}} o \textnf{\textit{Runge-Kutta}} 
para integrar la ecuación diferencial y obtener predicciones futuras del tamaño de la población.(\textbf{*aun no desarrollado en profundidad*})\\
\\
Para concluir notificamos que hemos identificado valores para confeccionar un dataset resultante de un completamiento de datasets con datos reales disponibles de población en nuestro pais(2), los cuales
incluyen el tamaño de la población en cada año desde el año 1980 hasta el 2022, cantidad de nacidos, fallecidos y tasa intrinseca de crecimiento poblacional. 
\\
\url{https://github.com/LFrench03/Proyecto-Numerica/blob/e24f82cfeb5900c173cc97d22633c309d6c9fd09/dataset.csv}
\\\textit{\textbf{\underline{Referencias}}:}
\begin{itemize}
    \item (1)Blog del Instituto de Matemáticas de la Universidad de Sevilla \\\url{https://institucional.us.es/blogimus/2018/01/un-ejemplo-sencillo-de-modelizacion-matematica-el-crecimiento-de-poblaciones/}
    \item (2)Sitio web de la Oficia Nacional de Estadísticas e Información(ONEI) en la sección de estadísticas demográficas \\ \url{https://www.onei.gob.cu/}
    \item (3)Tasa intrinseca de crecimiento poblacional \\\url{https://apuntesdedemografia.com/curso-de-demografia/temario/tema-3-crecimiento-y-estructura-de-la-poblacion/calculo-del-crecimiento-de-la-poblacion/}
    \item (4)Herramientas para proyectar la poblacion \\\url{https://ccp.ucr.ac.cr/cursos/icamacho/public_html/planificacion/contenido/tema6.htm}
\end{itemize}
\end{document}
