\documentclass[11pt, letterpaper]{article}
\usepackage[utf8]{inputenc}
\usepackage{hyperref}
\title{\textbf{\underline{Proyecto de Matemática Numérica}} \\ \textit{Predicción de crecimiento demográfico.} \\ \textit{Informe \#1}}
\author{Luis Ernesto Serras Rimada \\ Guillermo Cepero García \\ Miguel Vadim Vilariño Pedraza}

\date{\today}
\begin{document}
\maketitle
En nuestro proyecto final de \textit{Matemática Numérica}, nos gustaría trabajar en el problema de predecir el crecimiento
de una población en función del tiempo. Este es un problema importante en muchos campos, como la demografía, la
epidemiología y la ecología, que puede resultar útil en estudios demográficos, planificación urbana y análisis de recursos
naturales, entre otros. \\
Existen numerosos modelos matemáticos que se pueden utilizar para predecir el crecimiento de la población, 
como podría ser el \textit{\textbf{modelo logístico}}(1), una ecuacion diferencial que asume que la tasa
de crecimiento de la población es proporcional al tamaño de la población y a la capacidad de carga del entorno.
Para utilizar el modelo logístico para predecir el crecimiento de la población, se necesitan datos sobre el tamaño
de la población en diferentes momentos. Estos datos se pueden utilizar para estimar los parámetros del modelo y
luego utilizarlo para predecir el tamaño de la población en el futuro. Por lo cual hemos identificado un
conjunto de datos disponible de datos reales sobre el crecimiento de la población en nuestro pais(2), cuyos datos
incluyen el tamaño de la población en cada desde el año 1980 hasta el 2022. Luego, utilizaremos estos datos para estimar los parámetros del modelo logístico y luego emplear modelo numérico para 
predecir el tamaño de la población en los próximos n años de forma tal que el resultado converja lo máximo a la 
solución exacta.\\
\textit{\textbf{Ecuacion Diferencial de Modelo Logístico:}}\\
Donde P es la poblacion en función del tiempo(t) y r es la tasa de crecimiento de la población:
$\frac{dP}{dt} = r * P(t)$
\\
\begin{table}[t]
    \begin{center}
        \begin{tabular}{| c | c | c |}\hline \hline
            \textbf{Años} & \textbf{Densidad poblacional} & \textbf{Tasa(por cada 1000 personas)}}\\ \hline \hline
            1980 & 9 693 907 & -6.2\\
            1981 & 9 753 243 & 6.1\\
            1982 & 9 844 836 & 9.3\\
            1983 & 9 938 760 & 9.5\\
            \dots & \dots & \dots\\
            2022 & 11 089 511 & -2.1\\ \hline
        \end{tabular}
        \caption{Muestra en formato tabular de un dataset con datos sobre Densidad de Poblacional (1980-2022)(2)}
        \label{tab:años}
    \end{center}
\end{table}
\\ \\\textit{\textbf{Referencias}:}
\begin{itemize}
    \item (1)Blog del Instituto de Matemáticas de la Universidad de Sevilla \\\url{https://institucional.us.es/blogimus/2018/01/un-ejemplo-sencillo-de-modelizacion-matematica-el-crecimiento-de-poblaciones/}
    \item (2)Sitio web de la Oficia Nacional de Estadísticas e Información(ONEI) en la sección de estadísticas demográficas \\ \url{https://www.onei.gob.cu/}
\end{itemize}
\end{document}
